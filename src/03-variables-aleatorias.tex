\section{Variables Aleatorias}
\begin{enumerate}
	\setcounter{enumi}{15}
	\item
		Definición: $F_X(x)$ es la función que en cada punto vale $P(X\leq x)$.
		
		Propiedades:
			\begin{enumerate}
				\item $0 \leq F_X(x) \leq 1$
				\item Es creciente: $x< y \Rightarrow F_X(x) \leq F_X(y)$
				\item $\lim_{x\rightarrow -\infty}F_X(x) = 0$ y $\lim_{x\rightarrow +\infty}F_X(x) = 1$
				\item $F_X$ es continua a derecha.
			\end{enumerate}
			
		Demostraciones:
			\begin{enumerate}
				\item Vale porque en cada punto es una probabilidad.
				\item
					Sean $x,k > 0$. $$F_X(x+k) = P(X\leq x+k) = P((X\leq x) \lor (x<X\leq x+k)) = P(X\leq x) + P(x<X\leq x+k)$$
					porque ambos eventos son disjuntos. Luego $$F_X(x+k) = F_X(x) + P(x<X\leq x+k) \geq F_X(x)$$
				\item
					Empezamos por demostrar que $\lim_{x \rightarrow +\infty}F_X(x) = 1$.
					
					Sea $\{A_n\}$ una sucesión de eventos tal que $A_n = \{X\leq n\}$.
					Esta es una sucesión creciente (o sea que $A_n \subset A_{n+1}$), y además $\lim_{n\rightarrow +\infty}A_n = \{x \in\mathbb{R}\}$.
					
					$$\lim_{n\rightarrow +\infty}F_X(n) = \lim_{n\rightarrow +\infty}P(A_n) = P(\{x\in\mathbb{R}\}) = 1$$
					el segundo paso se puede hacer porque la sucesión es creciente.
					
					Ahora vamos a demostrar que $\lim_{x \rightarrow -\infty}F_X(x) = 0$.
					
					Sea $\{A_n\}$ una sucesión de eventos tal que $A_n = \{X\leq -n\}$.
					Esta es una sucesión decreciente (o sea que $A_n \supset A_{n+1}$), y además $\lim_{n\rightarrow +\infty}A_n = \emptyset $.
					
					$$\lim_{n\rightarrow -\infty}F_X(n) = \lim_{n\rightarrow +\infty}P(A_n) = P(\emptyset) = 0$$
					el segundo paso se puede hacer porque la sucesión es decreciente.
				\item
					Queremos ver que, dados $x_0\in\mathbb{R}$, $\epsilon > 0$, existe $\delta > 0$ tal que:
					$$x_0 < x < x_0 + \delta \Rightarrow F_X(x_0) - \epsilon < F_X(x) < F_X(x_0) + \epsilon$$
					
					La primera parte se cumple porque $F_X$ es una función creciente. Queda ver que $F_X(x) < F_X(x_0) + \epsilon$.
					
					Consideramos la sucesión $\{A_n\}$ tal que $A_n = \{X \in (-\infty, x_0 + 1/n]\}$. Sea $A = \{X \in (-\infty, x_0]\}$.
					Esta sucesión es decreciente y además $P(A_n) = F_X(x_0+\frac{1}{n})$.
					
					Además $\lim_{n\rightarrow \infty}A_n = A$
					
					$$\lim_{n\rightarrow \infty}F_X(x_0 + 1/n) = \lim_{n\rightarrow \infty}P(A_n) = P(A) = F_X(x_0)$$
					
					Entonces, como $F_X$ es creciente, dados $x_0, \epsilon$ existe $n_0$ tal que:
					$$n > n_0 \Rightarrow F_X(x_0 + 1/n) < F_X(x_0) + \epsilon$$
					
					Así que tomando $\delta = 1/n_0$ se cumple la condición.
			\end{enumerate}
	\item
		Sea $a = F_X(x_0) - P(X=x_0)$. Queremos ver que dados $x_0, \epsilon > 0$, existe $\delta > 0$ tal que
		$$x_0 - \delta < x < x_0 \Rightarrow a - \epsilon \leq F_X(x) \leq a + \epsilon$$
		
		Al mismo tiempo $a = P(X\leq x_0) - P(X=x_0) = P(X < x_0)$.
		Como $x < x_0$, se cumple que $P(X\leq x) \leq P(X<x_0)$. Entonces:
		$$F_X(x) = P(X\leq x) \leq P(X<x_0) = a$$
		Por lo tanto $F_X(x) \leq a + \epsilon$. Queda ver $a - \epsilon \leq F_X(x)$.
		
		Tomamos $\{A_n\}$ tal que $A_n=\{X < x_0-1/n\}$. Es una sucesión creciente y $\bigcup_{i=1}^{\infty}A_i = \{X<x_0\}$.
		
		Entonces $$\lim_{n\rightarrow\infty}F_X(x_0 - 1/n) = \lim_{n\rightarrow\infty}P(A_n) = P(X < x_0) = a$$
		
		Por lo tanto existe $n_0$ tal que $n>n_0\Rightarrow F_X(x_0 - 1/n_0) \geq a - \epsilon$.
		Como $F_X$ es creciente, tomando $\delta = 1/n_0$ se obtiene la propiedad. 
	\item
		
	\item
		
	\item
		
	\item
		
	\item
		
	\item
		
	\item
		
	\item
		
\end{enumerate}
