\section{Intervalos de Confianza}
\begin{enumerate}
	\setcounter{enumi}{114}
	\item
		TCL: Sea $X_1, \cdots, X_n$ i.i.d. de media $\mu$ y varianza $\sigma^2$. Entonces se cumple que:

		$$\frac{\sqrt{n}}{\sigma}(\overline X_n - \mu)\xrightarrow[n\rightarrow\infty]{d} N(0,1)$$

		Utilizando el TCL puede demostrarse que:
		$$Bi(n,p)\xrightarrow[n\rightarrow\infty]{d} N(np,np(p-1))$$

		Demostración:

		Dada $X\sim Bi(n,p)$, definamos a las $n$ pruebas independientes que conforman a $X$ como $Y_1, ... , Y_n$, con $Y_i \sim Ber(p)$. Es decir que

		$$\sum_{i=1}^{n}Y_i = X$$

		Como $Y_1, ... , Y_n$ son i.i.d. podemos aplicar el TCL y sabemos que:

		$$Z = \frac{\sqrt{n}}{\sqrt{p(1-p)}}(\frac{1}{n}\sum_{i=1}^{n}Y_i - p)\xrightarrow[n\rightarrow\infty]{d} N(0,1)$$

		Si $Z \xrightarrow{d} N(0,1)$, luego:

		$$Z\sqrt{np(1-p)}+np = \sum_{i=1}^{n}Y_i = X \xrightarrow[n\rightarrow\infty]{d} N(np,np(1-p))$$

		Porque si $Z \sim N(0,1)$ entonces $\sigma Z + \mu \sim N(\mu , \sigma^2)$

		Para ver cómo se arma el IC ver ej. 116.

	\item
		Sea $\hat{p_n}$ la proporción de éxitos en la muesta.
		Sea el pivote $$Z_n = \frac{\hat{p_n} - p}{\sqrt{p(1-p)}}\cdot \sqrt{n}$$

		Por LGN, cuando $n$ es grande, $Z_n\sim N(0,1)$.

		\begin{align*}
			P(|Z_n| < z_{1-\frac{\alpha}{2}})							& = 1-\alpha	\\
			P(-z_{1-\frac{\alpha}{2}} < Z_n < z_{1-\frac{\alpha}{2}})	& = 1-\alpha	\\
		\end{align*}
		$$-z_{1-\frac{\alpha}{2}} < Z_n < z_{1-\frac{\alpha}{2}}$$
		$$-z_{1-\frac{\alpha}{2}} <  \frac{\hat{p_n} - p}{\sqrt{p(1-p)}}\cdot \sqrt{n} < z_{1-\frac{\alpha}{2}}$$
		$$-z_{1-\frac{\alpha}{2}}\cdot \frac{\sqrt{p(1-p)}}{\sqrt{n}} <  \hat{p_n} - p < z_{1-\frac{\alpha}{2}}\cdot \frac{\sqrt{p(1-p)}}{\sqrt{n}}$$
		$$\hat{p_n} - z_{1-\frac{\alpha}{2}}\cdot \frac{\sqrt{p(1-p)}}{\sqrt{n}} < p < \hat{p_n} + z_{1-\frac{\alpha}{2}}\cdot \frac{\sqrt{p(1-p)}}{\sqrt{n}}$$

		Reemplazo $\sqrt{p(1-p)}$ por el peor caso.
		$$\hat{p_n} - z_{1-\frac{\alpha}{2}}\cdot \frac{1}{2\sqrt{n}} < p < \hat{p_n} + z_{1-\frac{\alpha}{2}}\cdot \frac{1}{2\sqrt{n}}$$
		Por lo tanto el IC es
		$$\left[\hat{p_n} - \frac{z_{1-\frac{\alpha}{2}}}{2\sqrt{n}}, \hat{p_n} + \frac{z_{1-\frac{\alpha}{2}}}{2\sqrt{n}}\right]$$
		\textbf{Relación entre $r$, $\alpha$ y $n$}
		$$r = \frac{z_{1-\frac{\alpha}{2}}}{2\sqrt{n}}$$
	\item
		Como $X_i\sim E(\lambda)$, $Y_i = 2\lambda X_i \sim E(1/2)$.
		Entonces:
		$$2\lambda\sum X_i = \sum Y_i \sim \Gamma(n, 1/2) \sim \chi^2_{2n}$$

		$$P(\chi^2_{2n, 1-\frac{\alpha}{2}} < 2\lambda\sum X_i < \chi^2_{2n, \frac{\alpha}{2}}) = 1 - \alpha$$
		$$P\left(\frac{\chi^2_{2n, 1-\frac{\alpha}{2}}}{2\sum X_i} < \lambda < \frac{\chi^2_{2n, \frac{\alpha}{2}}}{2\sum X_i}\right) = 1 - \alpha$$

		Entonces el IC es:
		$$\left[\frac{\chi^2_{2n, 1-\frac{\alpha}{2}}}{2\sum X_i}, \frac{\chi^2_{2n, \frac{\alpha}{2}}}{2\sum X_i}\right]$$
	\item
		Sea $X_1, X_2, \cdots, X_n \sim N(\mu, \sigma^2)$ una muestra de variables iid.
		$$Z_n = \frac{\sqrt{n}(\overline X - \mu)}{\sigma}$$
		Sabemos que $Z\sim N(0,1)$.

		Luego:
		\begin{align*}
			P(z_{\frac{\alpha}{2}} < Z_n < z_{1-\frac{\alpha}{2}})	& = 1 - \alpha	\\
			P\left(-z_{1-\frac{\alpha}{2}} < \frac{\sqrt{n}(\overline X - \mu)}{\sigma} < z_{1-\frac{\alpha}{2}}\right)	& = 1 - \alpha	\\
		\end{align*}
		$$-z_{1-\frac{\alpha}{2}} < \frac{\sqrt{n}(\overline X - \mu)}{\sigma} < z_{1-\frac{\alpha}{2}}$$
		$$-z_{1-\frac{\alpha}{2}}\cdot\frac{\sigma}{\sqrt{n}} < \overline X - \mu < z_{1-\frac{\alpha}{2}}\cdot\frac{\sigma}{\sqrt{n}}$$
		$$\overline X - z_{1-\frac{\alpha}{2}}\cdot\frac{\sigma}{\sqrt{n}} < \mu < \overline X + z_{1-\frac{\alpha}{2}}\cdot\frac{\sigma}{\sqrt{n}}$$
		Y el IC es:
		$$\left[\overline X - \frac{z_{1-\frac{\alpha}{2}}\sigma}{\sqrt{n}} ,  \overline X + \frac{z_{1-\frac{\alpha}{2}}\sigma}{\sqrt{n}}\right]$$
		(para la otra parte ver ejercicio 116).
	\item
		Sea $X_1, \cdots, X_n$ una muestra de VA iid con una distribución que depende de un parámetro $\theta$.
		Supongamos que existe $T(\underline X, \theta)$ función de la muestra y el parámetro que tiene una distribución que no depende de $\theta$.

		Si defino dos valores $a,b$ tales que $P(a \leq T(\underline X, \theta) < b) = 1-\alpha$, puedo construir un IC a partir de estos valores.

		\textbf{Ejemplo:}

		Sean $X_i\sim U[0,\theta]$. Sea $T = \frac{\text{máx}X_i}{\theta}$.
		Quiero averiguar la distribución de $T$.
		\begin{align*}
			P(T \leq a)	& = P\left(\frac{\text{máx}X_i}{\theta} \leq a\right) = P(\text{máx}X_i \leq a\theta)	\\
						& = \prod P(X_i \leq a\theta) = \prod F_{X_i}(a\theta) = F_{X_1}(a\theta)^n = \frac{(a\theta)^n}{\theta^n} = a^n
		\end{align*}
		Derivando:
		$$f_T(a) = n \cdot a^{n-1}$$
		Usando $T$ como pivote tenemos:
		\begin{align*}
			P(a\leq T \leq b)	& = 1-\alpha	\\
			P\left(a\leq \frac{\text{máx}X_i}{\theta}\leq b\right)	& = 1-\alpha	\\
			P\left(\frac{\text{máx}X_i}{b}\leq \theta\leq \frac{\text{máx}X_i}{a}\right)	& = 1-\alpha
		\end{align*}
		Entonces el IC es:
		$$\left[\frac{\text{máx}X_i}{b}, \frac{\text{máx}X_i}{a}\right]$$

		Se puede elegir cualquier par $(a,b)$ tal que se cumpla $P(a\leq T \leq b) = 1-\alpha$. O sea, $b^n - a^n = 1 - \alpha$.

		Como criterio, podemos minimizar la longitud esperada del intervalo.
		$$r = E(\text{máx} X_i) \left(\frac{1}{a} - \frac{1}{b}\right)$$
		O sea que queremos minimizar $$\frac{1}{a} - \frac{1}{b}$$

		$$b^n - a^n = 1 - \alpha\Rightarrow a = (\alpha - 1 +b^n)^{\frac{1}{n}}$$
		O sea que hay que minimizar:
		$$\frac{1}{(\alpha - 1 + b^n)^{\frac{1}{n}}} - \frac{1}{b}$$
		Que es decreciente (se verifica viendo que la derivada siempre es negativa). Luego $b=1$.
		$$a = (b^n - 1 + \alpha)^{1/n} = (1^n - 1 + \alpha)^{1/n} = \alpha^{1/n}$$
\end{enumerate}
