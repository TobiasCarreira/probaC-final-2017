\section{Vectores Aleatorios}
\begin{enumerate}
	\setcounter{enumi}{42}
	\item
		Sea $Y = g(X)$.
		\begin{align*}
			E(Y)	& = \sum_{y} y\cdot p(Y=y)								\\
					& = \sum_{y} \sum_{\{x : g(x) = y\}} y\cdot p(X=x)		\\
					& = \sum_{y} \sum_{\{x : g(x) = y\}} g(x)\cdot p(X=x)	\\
					& = \sum_{x} g(x)\cdot p(X=x)
		\end{align*}
		
		Cuando reemplazo $y=g(x)$, es importante que en cada término es potencialmente un $x$ distinto,
		pero como cada $x$ es de la preimagen de $g$ en $y$, las expresiones son equivalentes.
		
	\item
		\begin{enumerate}
			\item Caso continuo:
				$$E(X) = \int_0^{+\infty} x\cdot f_X(x)$$
				Hago partes usando el reemplazo $du = f_X(x)dx$, $u = -(1 - F_X(x))$ (o sea, hago aparecer lo que quiero en la integral de partes).
				
				Quedan $v = x$, $dv = dx$.
				\begin{align*}
					E(X)	& = \int_0^{+\infty} x\cdot f_X(x)	\\
							& = -x(1-F_X(x))\Big|_0^\infty - \int_0^{+\infty} -(1 - F_X(x))dx	\\
							& = -x(1-F_X(x))\Big|_0^\infty + \int_0^{+\infty} 1 - F_X(x)dx
				\end{align*}
				
				Queda ver que $x(1-F_X(x))\big|_0^\infty = 0$. Para $x=0$ se ve reemplazando, falta ver qué pasa cuando $x\rightarrow +\infty$.
				Pero,
				$$0 \leq x(1-F_X(x)) = x\int_x^{+\infty}f_X(u)du \leq \int_x^{+\infty}u\cdot f_X(u)du$$
				
				Como la esperanza está acotada, la parte de la derecha tiende a $0$ cuando $x\rightarrow +\infty$.
			\item Caso discreto:
				Sean $\{x_0, x_1, \cdots, x_n, \cdots\}$ los elementos del rango de $X$, en orden creciente. Sea $x_0=0$
				(si $0$ no está en el rango, lo agrego con probabilidad asociada $0$ porque si no está se me rompe todo con las geométricas).
				
				Se cumple que $p(x_i) = F(x_i) - F(x_{i-1})$.
				\begin{align*}
					E(X)	& = \sum_{i=1}^{+\infty} x_i\cdot{p(x_i)}				\\
							& = \sum_{i=1}^{+\infty} x_i\cdot(F(x_i) - F(x_{i-1}))	\\
							& = \sum_{i=1}^{+\infty} \sum_{j=1}^{i} (x_j - x_{j-1})\cdot(F(x_i) - F(x_{i-1}))		\\
							& = \sum_{j=1}^{+\infty} \sum_{i=j}^{+\infty} (x_j - x_{j-1})\cdot(F(x_i) - F(x_{i-1}))	\\
							& = \sum_{j=1}^{+\infty} (x_j - x_{j-1})\cdot(1 - F(x_{j-1}))							\\
				\end{align*}
				
				Lo último que quedó es la suma de cada rectangulito de la $F$ (que en el caso discreto es constante salvo en los saltos que ocurren en los $x_i$).
				Luego equivale a la integral de la función.
		\end{enumerate}

\end{enumerate}
