\section{Generación de Variables Aleatorias}
\begin{enumerate}
	\setcounter{enumi}{66}
	\item
        Definición de inversa generalizada ($F^{-1}:[0,1]\rightarrow \mathbb{R}$):
        $$F^{-1}(u) = \text{inf}\{x:F(x)\geq u\}$$

        Sea $U\sim U[0,1]$. Sea $Y = F_X^{-1}(U)$.
        $$F_Y(y) = P(F_X^{-1}(U) \leq y) = P(U\leq F_X(y)) = F_X(y)$$
        Luego $Y\sim X$.
	\item
		Sea $U\sim U[0,1]$.

		$X_1 = \mathbb{I}(u\leq p1)$ y $X_2 = \mathbb{I}(u\leq p2)$.
	\item
		Sea $U\sim U[0,1]$. $X_i = i\cdot U$. Se verifica que son uniformes y además que $X_i \leq X_{i+1}$.
	\item
        Sea $U\sim U[0,1]$. $X_k = -log(1-U)k$.
        $$F_{X_k}(x) = P(-log(1-U)k \leq x) = P(1-U \geq e^{-x/k}) = P(U < 1 - e^{-x/k}) = 1-e^{-x/k}$$
        Entonces tienen la distribución que se pide. Al mismo tiempo:
        $$-log(1-U)k \leq -log(1-U)(k+1)$$ porque $-log(1-U) \geq 0$. Luego la sucesión es creciente.
	\item
		\begin{align*}
			F_X(x)	& = P(X\leq x) = P\left(\frac{-log(1-U)}{\lambda} \leq x\right)	\\
					& = P\left(1-U \geq e^{-\lambda x}\right)	\\
					& = P(U \leq 1 - e^{-\lambda x}) = 1 - e^{-\lambda x}
		\end{align*}
		Entonces $X\sim \varepsilon(\lambda)$.
	\item
		$$P(Y_i \leq k | (U_i, Y_i) \in A) = \frac{P(Y_i \leq k \land (U_i, Y_i) \in T)}{P( (U_i, Y_i) \in T)}$$
		Miramos el numerador y el denominador por separado:
		\begin{align*}
			P(Y_i \leq k \land (U_i, Y_i) \in T)	& = P\left(Y_i \leq k \land U_i \leq \frac{f(Y_i)}{c\cdot g(Y_i)}\right)		\\
									& = \int_{-\infty}^{k}\int_{0}^{\frac{f(y)}{c\cdot g(y)}} f_{Y_i,U_i}(y,u) du\text{ }dy			\\
									& = \int_{-\infty}^{k}\int_{0}^{\frac{f(y)}{c\cdot g(y)}} f_{Y_i}(y)f_{U_i}(u) du\text{ }dy		\\
									& = \int_{-\infty}^{k} f_{Y_i}(y) \int_{0}^{\frac{f(y)}{c\cdot g(y)}} f_{U_i}(u) du\text{ }dy	\\
									& = \int_{-\infty}^{k} g(y)\cdot F_{U_i}\left(\frac{f(y)}{c\cdot g(y)}\right) dy				\\
									& = \int_{-\infty}^{k} g(y)\cdot \frac{f(y)}{c\cdot g(y)} dy									\\
									& = \int_{-\infty}^{k} \frac{f(y)}{c} dy = \frac{F_X(k)}{c}
		\end{align*}
		El denominador:
		\begin{align*}
			P((U_i, Y_i) \in T)	& = P\left(U_i \leq \frac{f(Y_i)}{c\cdot g(Y_i)}\right)	\\
								& = \int_{-\infty}^{+\infty}\int_{0}^{\frac{f(y)}{c\cdot g(y)}} f_{Y_i,U_i}(y,u) du\text{ }dy		\\
								& = \int_{-\infty}^{+\infty}\int_{0}^{\frac{f(y)}{c\cdot g(y)}} f_{Y_i}(y)f_{U_i}(u) du\text{ }dy	\\
								& = \int_{-\infty}^{+\infty}g(y)\cdot F_{U_i}\left(\frac{f(y)}{c\cdot g(y)}\right) dy				\\
								& = \int_{-\infty}^{+\infty}\frac{f(y)}{c} dy = \frac{1}{c} \int_{-\infty}^{+\infty}f(y) dy	= \frac{1}{c}
		\end{align*}
		Combinando ambas expresiones queda:
		$$P(Y_i \leq k | (U_i, Y_i) \in A) = \frac{F_X(k)}{c} \frac{1}{1/c} = F_X(k)$$
		En particular vale que $i=T \Rightarrow (U_i, Y_i) \in A$.
		Entonces $F_T(k) = F_X(k)$.
\end{enumerate}
