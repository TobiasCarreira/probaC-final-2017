\section{Coupon collector}
\begin{enumerate}
	\setcounter{enumi}{130}
	\item
		Voy a asumir que empiezo a contar desde 0 y no desde 1 porque me hace feliz, y los índices quedan más lindos.
	
		Primero, $T_i $ es el número de intentos hasta el primer éxito (encontrar un elemento nuevo), con una probabilidad de éxito de:
		$$\frac{\text{elementos nuevos}}{\text{elementos totales}} = \frac{n-i}{n}$$ Además, las $T_i$ son independientes.
		
		En particular, $T_0$ (el tiempo hasta el primer elemento) tiene una probabilidad de éxito de 1.
		
		Esta definición de $T_i$ quiere decir que $T_i \sim G(\frac{n-i}{n})$. Por esta razón, $E(T_i) = \frac{n}{n-i}$. 		
		$$E(T) = \sum_{i=0}^{n-1}E(T_i)$$
		$$E(T) = \sum_{i=0}^{n-1} \frac{n}{n-i}$$
		$$E(T) = n \cdot \sum_{i=0}^{n-1} \frac{1}{n-i}$$
		$$E(T) = n \cdot \sum_{i=1}^{n} \frac{1}{i}$$
		
		FALTA TERMINAR
\end{enumerate}
